\documentclass[12pt,a4paper]{scrartcl}
\usepackage[utf8]{inputenc}
\usepackage[english,russian]{babel}
\usepackage{indentfirst}
\usepackage{misccorr}
\usepackage{graphicx}
\usepackage{amsmath}
\usepackage{leftidx}
\usepackage{tikz}
\usetikzlibrary{shapes,arrows}
\begin{document}

\section*{Материалы к экзамену}

\subsection*{Именные теоремы}

\textbf{Бэр (о полном метрическом пространстве)}: Полное метрическое пространство не может быть представлено в виде счетного объединения своих нигде не плотных множеств. \\

\textbf{Хаусдорф (о пополнении)}: Для любого неполного метрического пространства существует его пополнение. \\
 
\textbf{Александер (о предбазе)}: \\

\textbf{Тихонов (о компактности декартова произведения компактных ТП)}: \\

\textbf{Критерий Фреше (топологическая компактность в МП)}: $(X, \rho)$ — МП, подмножество $S \subset X.$ Следующие 3 свойства эквивалентны: \\
1) $S$ — компактное множество;
2) $S$ — вполне ограниченное множество и $(S, \rho)$ — полное метрическое пространство;
3) $S$ — секвенциально компактное множество.\\

\textbf{Арцела, Асколи}: $(K, \rho)$ — компактное МП, $C(K)$ — пространство непрерывных на компакте $K$ функций, $S \subset C(K).$ Множество $S$ является вполне ограниченным множеством тогда и только тогда, когда: \\
1) $S$ — ограниченное множество;\\
2) $S$ — равностепенно непрерывное множество.\\

\textbf{Рисс, Колмогоров}: \\

\textbf{Рисс (лемма о почти перпендикуляре)}: $(X, \Vert \cdot \Vert_X)$ — ЛНП, $L \subset X$ — собственное замкнутое подпространство. Тогда 
$$\forall \varepsilon in (0, 1) \ \ \exists z_{\varepsilon}: \Vert z_{\varepsilon}\Vert_X = 1, \rho(z_{\varepsilon}, L) > 1 - \varepsilon.$$\\

\textbf{Рисc (о некомпактности сферы)}: В бесконечномерном ЛНП единичная сфера не является компактным множеством. [простое следствие леммы о почти перпендикуляре]\\

\textbf{Банах (об открытом отображении)}: $(X, \Vert \cdot \Vert_X), (Y, \Vert \cdot \Vert_Y)$ — банаховы пространства, $A \in L(X, Y)$ — сюръективный оператор. Тогда $A$ является открытым отображением. \\

\textbf{Банах (об обратном операторе)}: $(X, \Vert \cdot \Vert_X), (Y, \Vert \cdot \Vert_Y)$ — банаховы пространства, $A \in L(X, Y).$ Оператор $A$ непрерывно обратим $A^{-1} \in L(Y, X)$ тогда и только тогда, когда $\begin{cases} Ker A = \{0\} \\ Im A = Y\end{cases}.$\\

\textbf{Хан, Банах} (реально это частный случай более общего утверждения): $(X, \Vert \cdot \Vert_X)$ — комплексное ЛНП, $L \subset X$ — подпространство, $f: L \rightarrow \mathbb{C}$ — линейный непрерывный функционал на $L.$ Тогда существует линейный непрерывный функционал $g: X \rightarrow \mathbb{C}$ на $X$ такой, что: \\
1) $f\Bigm|_{L} = g\Bigm|_{L};$\\
2) $\Vert f\Vert = \Vert g\Vert.$\\
\textbf{4 следствия теоремы Хана-Банаха}: \\

\textbf{Рисс (о проекции)}: $H$ — ГП, $S \subset H$ — выпуклое замкнутое множество. Тогда для любого элемента ГП существует и единствена метрическая проекция этого элемента на множество $S:$
$$\forall x \in H \ \ \exists ! y \in S: \rho(x, S) = \Vert x - y\Vert.$$

\textbf{Рисс (об ортогональном дополнении)}: $H$ — ГП, $L \subset H$ — замкнутое подпространство. Тогда 
$$H = L \oplus L^{\bot}.$$\\ 

\textbf{Рисс, Фреше}: $H$ — ГП, $f \in H^*$ — линейный непрервный функционал на $H.$ Тогда существует и единственен $z_f \in H$ такой, что: \\
1) $\forall x \in H: f(x) = (x, z_f);$\\
2) $\Vert f\Vert = \Vert z_f\Vert_H.$\\
Отображение $z: H^* \rightarrow H$ является взаимооднозначным, изометричным и сопряженно-линейным.\\
Следствием этой теоремы является \textbf{рефлексивность произвольного ГП}.\\

\textbf{Шур} В пространстве $(l_1, \Vert \cdot \Vert_1)$ всякая слабо сходящаяся последовательность сходится сильно.\\

\textbf{Мазур}: $(X, \Vert \cdot \Vert)$ — ЛНП. Пусть $A \subset X$ — выпуклое множество. Тогда мн-во $A$ слабо замкнуто тогда и только тогда, когда оно сильно замкнуто. \\

\textbf{Банах, Алаоглу}: $(X, \Vert \cdot \Vert)$ — ЛНП. Пусть $(X, \Vert \cdot \Vert)$ является сепарабельным пространством. Тогда любой замкнутый шар в сопряженном пространстве $\forall R > 0 \ \ B_R^{*}(0) \subset X^{*}$ является слабо* компакнтым. \\

\textbf{Банах, Тихонов}: (является следствием теоремы Банаха, Алаоглу) $(X, \Vert \cdot \Vert)$ — ЛНП. Пусть $(X, \Vert \cdot \Vert)$ является сепарабельным и рефлексивным пространством. Тогда любая ограниченная последовательность в пространстве $(X, \Vert \cdot \Vert)$ содержит слабо сходящуюся подпоследовательность. \\

\textbf{Банах, Штейнгауз}: $(X, \Vert \cdot \Vert_X), (Y, \Vert \cdot \Vert_Y)$ — ЛНП, причем $(X, \Vert \cdot \Vert_X)$ банахово пространство. Пусть последовательность линейных непрерывных операторов $\left\{A_n\right\}_{n \in \mathbb{N}} \subset L(X, Y)$ является поточечно ограниченной: $\forall x \in X \sup\limits_{n \in \mathbb{N}}\Vert A_nx\Vert_Y < +\infty.$ Тогда она является ограниченной в пространстве $L(X, Y):\sup\limits_{n \in \mathbb{N}}\Vert A_n\Vert < \infty.$\\
Следствием этой теоремы служит ограниченность слабо и слабо* сходящихся последовательностей.\\

\textbf{Фредгольм}: $(X, \Vert \cdot \Vert_X), (Y, \Vert \cdot \Vert_Y)$ — ЛНП, $A \in L(X, Y).$ Тогда
$$Ker A =  ^{\perp}(Im A^{*}), Ker A^{*} = (Im A)^{\bot}.$$
\textbf{Следствие}: $(X, \Vert \cdot \Vert_X), (Y, \Vert \cdot \Vert_Y)$ — ЛНП, $A \in L(X, Y).$ Тогда 
$$(Ker A)^{\bot} =  (^{\perp}(Im A^{*}))^{\bot} \supset [Im A^{*}], ^{\perp}(Ker A^{*}) = ^{\perp}((Im A)^{\bot}) = [Im A].$$
Если $X$ — рефлексивно, то 
$$(Ker A)^{\bot} =  (^{\perp}(Im A^{*}))^{\bot} = [Im A^{*}].$$

\textbf{Гельфанд, Мазур}: $A$ — банахова алгебра, в которой каждый ненулевой элемент обратим. Тогда $A$ изометрически изоморфна полю комплексных чисел. \\

\textbf{Хелингер, Теплиц}: $H$ — ГП, $A$ — линейный симметричный оператор на всем $H.$ Тогда $A$ является непрерывным, а следовательно самосопряженным оператором. \\

\textbf{Гильберт, Шмидт}: \\

\textbf{Спектральная теорема}: \\

\newpage
\subsection*{Основные определения}

\subsubsection*{Топологическое пространство}

\subsubsection*{Метрическое пространство}

\subsubsection*{Нормированное пространство}

\subsubsection*{Гильбертово пространство}

\subsubsection*{Компактность}

\subsubsection*{Виды операторов}

\textbf{Ограниченный оператор}: \\

\textbf{Норма оператора}:\\

\textbf{Ограниченный снизу оператор}: \\

\textbf{Обратный оператор}: \\

\textbf{Компактный оператор}: \\

\textbf{Сопряженный оператор}: \\ 

\textbf{Нормальный оператор}: \\ 

\textbf{Унитарный оператор}: \\ 

\textbf{Оператор проекции}: \\ 

\textbf{Симметричный оператор}: \\ 

\textbf{Самосопряженный оператор}: \\

\textbf{Неотрицательный оператор}: \\ 

\subsubsection*{Спектр}

\textbf{Спектр}: \\ 

\textbf{Резольвентное множество}: \\ 

\textbf{Резольвента}: \\ 

\textbf{Спектральный радиус}: \\ 

\textbf{Точечный спектр оператора}: \\ 

\textbf{Непрерывный спектр оператора}: \\ 

\textbf{Остаточный спектр оператора}: \\ 

\newpage
\subsection*{Важные теоремы}

\textbf{Критерий базы ТП} $X$ — множество. Семейство $\beta$ подмножеств множества $X$ является базой некоторой топологии $\tau$ в $X$ тогда и только тогда, когда: \\
1) $\forall x \in X: \exists V \in \beta: x \in V, \text{ то есть }X = \bigcup\limits_{V \in \beta} V;$\\
2) $\forall V_1, V_2 \in \beta \ \ \forall x \in V_1 \cap V_2 \ \ \exists W \in \beta: x \in W \subset V_1 \cap V_2.$\\

\textbf{Критерий несепарабельности МП} Метрическое пространство является несепарабельным тогда и только тогда, когда в нем существует более чем счетное $\varepsilon_0 > 0$ - дырявое множество. \\

\textbf{Принцип вложенных шаров} Метрическое пространство является полным тогда и только тогда, когда произвольная последовательность замкнутых вложенных шаров, чьи радиусы стремятся к нулю, имеет непустое пересечение. \\
\textbf{Контрпример}. Полное МП и последовательность замкнутых вложенных шаров, радиусы которых не стремятся к нулю, имеет пустое пересечение. \\

\textbf{Теорема о полноте пространства операторов} $(X, \Vert \cdot \Vert_X), (Y, \Vert \cdot \Vert_Y)$ — ЛНП, причем $(Y, \Vert \cdot \Vert_X)$ банахово пространство. Тогда $(L(X, Y), \Vert \cdot\Vert_A)$ является банаховым пространством. \\
Тривиальным следствием этого факта является банаховость произвольного сопряженного пространства, так как $X^* = L(X, \mathbb{C}).$\\

\newpage
\subsection*{Схема определения частей спектра линейного непрерывного оператора}

\tikzstyle{decision} = [rectangle, draw, fill=green!20,  text width=12em, text centered, rounded corners, minimum height=3em]
\tikzstyle{block} = [rectangle, draw, fill=blue!20, 
    text width=5em, text centered, rounded corners, minimum height=4em]
\tikzstyle{line} = [draw, -latex']
\tikzstyle{cloud} = [draw, rectangle,fill=red!20,  text centered, text width=12em, node distance=3cm,
    minimum height=3em]

\begin{tikzpicture}[node distance = 2cm, auto]
    % Place nodes
    \node [block] (complex) {$\lambda \in \mathbb{C}$};
    \node [cloud, below left of=complex, node distance=5cm] (resolvent) {$Ker A_\lambda = \{0\}, Im A_\lambda = X$};
    \node [cloud, below right of=complex, node distance=5cm] (spectrum) {$Ker A_\lambda \neq \{0\}$ или $Im A_\lambda \neq X$};
    \node [decision, below of=resolvent] (resolvent_how) {$A_\lambda x = 0 \Leftrightarrow x = 0 \ \ \forall y \in X: A_{\lambda}x = y,$ т.е $x = R_A(\lambda)y$};
    \node [block, below of=resolvent_how] (resolvent_two) {$\lambda \in \rho(A)$};
    \node [block, below of=spectrum] (spectrum_two) {$\lambda \in \sigma(A)$};
    \node [cloud, below left of=spectrum_two, node distance=3cm] (pointer) {$Ker A_\lambda \neq \{0\}$};
    \node [decision, below of=pointer] (pointer_how) {$\exists x \neq 0: A_\lambda x = 0$};
    \node [block, below of=pointer_how] (pointer_two) {$\lambda \in \sigma_p(A)$};
    \node [cloud, below right of=spectrum_two, node distance=5cm] (other) {$Ker A_\lambda = \{0\}, Im A_\lambda \neq X$};
    \node [decision, below of=other, node distance=2cm] (other_how) {$A_\lambda x = 0 \Leftrightarrow x = 0 \ \ \exists y \in X: A_\lambda x \neq y \ \ \forall x \in X$};
    \node [cloud, below left of=other_how, node distance=5cm] (continious) {$Ker A_\lambda = \{0\}, Im A_\lambda \neq X, [Im A_\lambda] = X$ или $\lambda \in \sigma(A), Ker A_\lambda = \{0\}, Ker (A_\lambda)^* = \{0\}$};
    \node [block, below of=continious, node distance=2.5cm] (continious_two) {$\lambda \in \sigma_c(A)$};
    \node [cloud, below right of=other_how, node distance=3cm] (residual) {$Ker A_\lambda = \{0\}, [Im A_\lambda] \neq X$ или $\lambda \in \sigma(A), Ker A_\lambda = \{0\}, Ker (A_\lambda)^* \neq \{0\}$};
    \node [block, below of=residual, node distance=2.5cm] (residual_two) {$\lambda \in \sigma_r(A)$};
    % Draw edges
    \path [line] (complex) -- (resolvent);
    \path [line] (complex) -- (spectrum);
    \path [line] (resolvent) -- (resolvent_how);
    \path [line] (resolvent_how) -- (resolvent_two);
    \path [line] (spectrum) -- (spectrum_two);
    \path [line] (spectrum_two) -- (pointer);
    \path [line] (spectrum_two) -- (other);
    \path [line] (pointer) -- (pointer_how);
    \path [line] (pointer_how) -- (pointer_two);
    \path [line] (other) -- (other_how);
    \path [line] (other_how) -- (continious);
    \path [line] (other_how) -- (residual);
    \path [line] (continious) -- (continious_two);
    \path [line] (residual) -- (residual_two);
    
\end{tikzpicture}

\newpage
\subsection*{Известные операторы}

\end{document}